\documentclass{article}
\usepackage[utf8]{inputenc}
\usepackage{amsmath}
\usepackage{amssymb}
\usepackage{geometry}
\usepackage{circuitikz}
\usepackage{multicol}
\usepackage{titlesec}
\usepackage{parskip}

\geometry{a4paper, margin=0.5cm}
\fontsize{8pt}{0}



\title{Analogelektronik}

\author{Michael Korn}
\titlespacing*{\section}{0pt}{0.1\baselineskip}{0.2\baselineskip}
\begin{document}
\maketitle
\begin{multicols}{4}
    Ohmsches Gesetzt
    \begin{align*}
        U &= R \cdot I \\
        R &= \frac{U}{I} \\
        I &= \frac{U}{R} \\
        R &= \frac{1}{G} 
    \end{align*}
    
    Leistung
    \begin{align*}
        P &= U \cdot I \\
        P &= R \cdot I^2 \\
        P &= \frac{U^2}{R}
    \end{align*}
    
    Regeln
    \begin{align*}
        Knoten \\
        \sum I_{in} = \sum I_{out} \\
        Maschen \\
        \sum U = 0
    \end{align*}

    RC-Tiefpass
    \begin{align*}
        H &= \frac{1}{\sqrt{1+(\frac{f}{f_c})^2}} \\
        f_c &= \frac{1}{2\pi RC} \\
        H[db] &= 20\cdot log(H) \\
        -3db &= \frac{1}{\sqrt{2}}\cdot H
    \end{align*}
    
\end{multicols} 

\section*{Halbleiter}
\begin{multicols}{3}
    Materialspezifisch
    \begin{align*}
        R &= \rho \cdot \frac{l}{A} \\
        G &= \sigma \cdot \frac{A}{l}
    \end{align*}

    Elekonen \& Loecher
    \begin{align*}
        n_0 &= p_0 = n_i \\
        n_i &= n_{i0}(\frac{T}{T_0})^{\frac32}e^{\frac{W_g}{2kT_0}}e^{\frac{W_g}{2kT}} \\
        n&+N_A = p+N_D
    \end{align*}

    Donatoren \& Akzeptoren
    \begin{align*}
        n\cdot p &= n_i^2 \\
        n_p\cdot p_n &= n_i^2 \\
        n_n\cdot p_n &= n_i^2 
    \end{align*}

\end{multicols} 
\subsection*{Diffusion}
\begin{minipage}[t]{0.451\textwidth}
    \begin{equation*}
        j_{Feld}=e\cdot n?p?\cdot \nu=e\cdot n\cdot \mu\cdot E=\sigma\cdot E
    \end{equation*}
    \begin{equation*}
        j_{Diff} = e\cdot D\cdot \frac{dn?dp?}{dx}
    \end{equation*}
\end{minipage}
\begin{minipage}[t]{0.49\textwidth}
    \begin{equation*}
        I=A\cdot e\cdot E\cdot (n\cdot\mu_n+p\cdot\mu_p)
    \end{equation*}
    \begin{equation*}
        D=\mu\cdot\frac{k_BT}{e}
    \end{equation*}
    \begin{equation*}
        j_{Ges} = j_{Feld}+j_{Diff}=e\cdot D\cdot\frac{dn?dp?}{dx}+e\cdot n?p?\cdot\mu\cdot E
    \end{equation*}
\end{minipage}
Dabei bezeichnen:
\begin{multicols}{3}
\begin{itemize}
    \item $\mu$: Ladungsträgerbeweglichkeit
    \item $\sigma$: spezifische Leifähigkeit
    \item $\nu$: Driftgeschwindigkeit
    \item $n?p?$: Ladungsträgerdichte
    \item $D$: Diffusionskonstante für Locher oder Elektronen
\end{itemize}
\end{multicols}

\section*{Dioden}
\begin{minipage}[t]{0.4\textwidth}
\begin{equation*}
    I_D = I_S \cdot (e^{\frac{U}{n \cdot U_T}} - 1)
\end{equation*}
\end{minipage}
\begin{minipage}[t]{0.5\textwidth}
\begin{equation*}
    I_S = I_S(T_0)\cdot e^{(\frac{T}{T_0}-1)\cdot \frac{U_g}{n\cdot U_T}}\cdot (\frac{T}{T_0})^{\frac{x}{n}}
\end{equation*}
\end{minipage}

\begin{minipage}[t]{0.4\textwidth}
\begin{equation*}
    U_G=\frac{W_G}{e}=1.12V
\end{equation*}
\end{minipage}
\begin{minipage}[t]{0.5\textwidth}
    \begin{equation*}
        g_d=\frac{I}{n\cdot U_T}=\frac1{r_D}
    \end{equation*}
    \end{minipage}
wobei:
\begin{itemize}
    \begin{multicols}{3}
    \item $x\approx3$
    \item $U_G$: Bandlückens
    \item $I_D$: Diodenstrom
    \item $I_S$: Sättigungsstrom
    \item $n$: Emissionskoeffizient
    \item $U_T = \frac{k \cdot T}{q}$: Temperaturspannung
    \item $\approx I_S \cdot (e^{\frac{U}{n \cdot U_T}})$\\ für $U>4\cdot U_T$
    \item $g_d$: Kleinsignalleitwert
    \end{multicols} 
\end{itemize}

\begin{minipage}[t]{0.6\textwidth}
    \subsection*{Temperaturspannung}
    \begin{equation*}
        U_T = \frac{k_B \cdot T}{e} \approx 26 \, \text{mV bei } T = 300 \, \text{K}
    \end{equation*}
    wobei:
    \begin{itemize}
        \begin{multicols}{2}
        \item $k$: Boltzmann TR:25 in J
        \item $k_B$: $8.617\cdot 10^{-5}\frac{eV}{K}$
        \item $T$: Absolute Temperatur in Kelvin
        \item $1J=6.2415\cdot 10^{18}eV$
        \end{multicols}
    \end{itemize}
\end{minipage}
\begin{minipage}[t]{0.3\textwidth}
    \subsection*{weitere Dioden}
    \begin{itemize}
        \item Zener
        \item Schoty
        \item Fotodiode
        \item LED
    \end{itemize}

\end{minipage}

\section*{Bipolare Transistoren (BJT)}
\begin{minipage}[t]{0.5\textwidth}
    \subsection*{Stromverhältnisse}
        \begin{align*}
            I_C &= \beta \cdot I_B \\
            I_E &= I_C + I_B \\
            I_E &= (\beta + 1) \cdot I_B
        \end{align*}
    
\end{minipage}
\begin{minipage}[t]{0.5\textwidth}
\subsection*{Kleinsignal}
    \begin{align*}
        g_m &= \frac{I_C}{U_T} \\
        r_\pi &= \frac{\beta}{g_m} \\
        r_o &= \frac{U_A}{I_C}
    \end{align*}
    wobei:
    \begin{itemize}
        \item $g_m$: Steilheit
        \item $r_\pi$: Basis-Emitter-Widerstand
        \item $r_o$: Ausgangswiderstand
        \item $U_A$: Early-Spannung
    \end{itemize}
\end{minipage}
\subsection*{Verstärkung in verschiedenen Schaltungen}
\begin{itemize}
    \item \textbf{Emitterschaltung:} $V_u \approx -g_m \cdot R_C$ (bei großem $r_o$ und $R_E = 0$)
    \item \textbf{Kollektorschaltung (Emitterfolger):} $V_u \approx 1$
    \item \textbf{Basisschaltung:} $V_u \approx g_m \cdot R_C$
\end{itemize}
wobei $R_C$ der Kollektorwiderstand ist.

\subsection*{Early-Effekt}
Steilheit im aktiven Bereich
\begin{equation*}
    \frac{I_c}{U_{EA}}
\end{equation*}
\section*{MOSFETs}

\subsection*{Grundlagen}
MOSFETs gibt es in zwei Haupttypen:
\begin{itemize}
    \item \textbf{n-Kanal MOSFET:} Leitet bei positiver Gate-Source-Spannung.
    \item \textbf{p-Kanal MOSFET:} Leitet bei negativer Gate-Source-Spannung.
\end{itemize}

\subsection*{Strom-Spannungs-Beziehungen}
\begin{itemize}
    \item \textbf{Abschnitt (Cut-off):} $I_D = 0$ für $V_{GS} < V_{TH}$
    \item \textbf{Linearer Bereich (Triodenbereich):} $I_D = \mu_n C_{ox} \frac{W}{L} [(V_{GS} - V_{TH})V_{DS} - \frac{1}{2}V_{DS}^2]$ für $V_{GS} > V_{TH}$ und $V_{DS} < V_{GS} - V_{TH}$
    \item \textbf{Sättigungsbereich (Aktivbereich):} $I_D = \frac{1}{2} \mu_n C_{ox} \frac{W}{L} (V_{GS} - V_{TH})^2$ für $V_{GS} > V_{TH}$ und $V_{DS} \ge V_{GS} - V_{TH}$
\end{itemize}
wobei:
\begin{itemize}
    \item $\mu_n$: Beweglichkeit der Elektronen
    \item $C_{ox}$: Kapazität des Gate-Oxids pro Fläche
    \item $W$: Breite des Kanals
    \item $L$: Länge des Kanals
    \item $V_{TH}$: Schwellenspannung
\end{itemize}

\subsection*{Kleinsignalparameter}
\begin{itemize}
    \item \textbf{Steilheit:} $g_m = \frac{\partial I_D}{\partial V_{GS}} = \mu_n C_{ox} \frac{W}{L} (V_{GS} - V_{TH})$ (im Sättigungsbereich)
\end{itemize}

\section*{Operationsverstärker (OPV)}

\subsection*{Idealer OPV}
\begin{itemize}
    \item Unendlich hohe Eingangs-Impedanz
    \item Null Ausgangs-Impedanz
    \item Unendlich hohe Verstärkung
\end{itemize}

\subsection*{Realer OPV}
\begin{itemize}
    \item \textbf{Eingangsruhestrom (Bias Current):} $I_B$
    \item \textbf{Eingangs-Offsetspannung:} $V_{OS}$
    \item \textbf{Verstärkung mit offenem Regelkreis (Open-Loop Gain):} $A_{OL}$ (endlich)
    \item \textbf{Gleichtaktunterdrückung (CMRR)} 
    \item Common-Mode Rejection Ratio :Verhältnis der Diffverstärkung zur Gleichtaktverstärkung.
    \item \textbf{Bandbreite:} Frequenzbereich, in dem die Verstärkung des OPVs nicht wesentlich abfällt.
    \item \textbf{Slew Rate:} Maximale Änderungsgeschwindigkeit der Ausgangsspannung.
\end{itemize}


\subsection*{Invertierender Verstärker}

\begin{equation*}
    V = -\frac{R_2}{R_1}
\end{equation*}

\subsection*{Nicht-invertierender Verstärker}

\begin{equation*}
    V = 1 + \frac{R_2}{R_1}
\end{equation*}

\subsection*{Weitere wichtige OPV-Schaltungen}
\begin{itemize}
    \item \textbf{Differenzverstärker:} $V_{out} = \frac{R_2}{R_1}(V_2 - V_1)$ (wenn $R_1 = R_3$ und $R_2 = R_4$)
    \item \textbf{Integrierer:} $V_{out}(t) = -\frac{1}{RC} \int V_{in}(t) dt$
    \item \textbf{Differenzierer:} $V_{out}(t) = -RC \frac{dV_{in}(t)}{dt}$
\end{itemize}

\section*{Wechselstromanalyse (AC)}

\subsection*{Impedanz}
\begin{align*}
    Z_R &= R \\
    Z_C &= \frac{1}{j\omega C} \\
    Z_L &= j\omega L
\end{align*}
wobei:
\begin{itemize}
    \item $\omega = 2\pi f$: Kreisfrequenz
    \item $j$: Imaginäre Einheit
\end{itemize}

\end{document}